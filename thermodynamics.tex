\documentclass[a4paper,11pt]{jsarticle}

%%%
% packages
%%%
% 数式
\usepackage{amsmath,amsfonts,physics}
\usepackage{bm}
% 画像
\usepackage[dvipdfmx]{graphicx}
% 単位
\usepackage{siunitx}
\AtBeginDocument{\RenewCommandCopy\qty\SI}  % qty が競合

%%%
% マクロ
%%%
% 太字
\newcommand{\bx}{\vb*{x}}
\newcommand{\bb}{\vb*{b}}
\newcommand{\bc}{\vb*{c}}
% 単位
\DeclareSIUnit{\atm}{atm}
\newcommand{\braunit}[1]{\;\text{[}\unit{#1}\text{]}\;}


%%%
% 本文
%%%
\begin{document}

\title{熱力学メモ}
\author{NAKATA Keisuke}
\date{20240526}
\maketitle


\part*{熱力学}

\subsection*{アボガドロ定数 $N_A$}
\begin{align*}
  N_A \braunit{\per\mol}= \num{6.02214076e23}
\end{align*}
0.012 kg の炭素に含まれる炭素原子の数。 $N_A = \qty{1}{[\mol]}$. \\
\cite[pp.1]{thermo}

\subsection*{大気圧 $\qty{1}{[\atm]}$}
\begin{align*}
  \qty{1}{[\atm]} = \qty{1.01325e5}{[\Pa]}
\end{align*}
海面上で空気から受ける圧力の値。 \\
\cite[pp.1]{thermo}

\subsection*{絶対温度 $\unit{\K}$}
\begin{align*}
  T \braunit{\K} = t \braunit{\celsius} + 273.15
\end{align*}
\cite[pp.10]{thermo}

\subsection*{シャルルの法則}
\begin{align*}
  \varDelta V\braunit{\m^3}
    &= \varDelta t\braunit{\celsius}\times\frac{V_{\qty{0}{\celsius}}\braunit{\m^3}}{273.15} \\
    &= \varDelta T\braunit{\K}\times\frac{V_{\qty{273.15}{\K}}\braunit{\m^3}}{273.15}
\end{align*}
温度と体積の比例則。$V_{\qty{0}{\celsius}} (=V_{\qty{273.15}{\K}})$ は気体の種類や圧力、モル数によるので注意。高圧または低温では分子間力や分子の大きさを無視できず、近似が悪化する。 \\
\cite[pp.10]{thermo}

\subsection*{ボイル・シャルルの法則 (気体の状態方程式)}
\begin{align*}
  p\braunit{\Pa}V\braunit{\m^3} = n\braunit{\mol}RT\braunit{\K}
\end{align*}
$R$ は気体定数。シャルルの法則、ボイルの法則、ゲイ=リュサックの法則を組み合わせたもの。高圧または低温では分子間力や分子の大きさを無視できず、近似が悪化する。\\
\cite[pp.15]{thermo}

\bibliographystyle{jplain}
\bibliography{citations}

\end{document}
