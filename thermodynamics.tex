\documentclass[a4paper,11pt]{jsarticle}

%%%
% packages
%%%
% 数式
\usepackage{amsmath,amsfonts,physics}
\usepackage{bm}
% 画像
\usepackage[dvipdfmx]{graphicx}
% 単位
\usepackage{siunitx}
\AtBeginDocument{\RenewCommandCopy\qty\SI}  % qty が競合
% 参照機能の拡張
\usepackage[bookmarksnumbered=true]{hyperref}


%%%
% マクロ
%%%
% 太字
\newcommand{\bx}{\vb*{x}}
\newcommand{\bb}{\vb*{b}}
\newcommand{\bc}{\vb*{c}}
% 単位
\DeclareSIUnit{\atm}{atm}
\newcommand{\braunit}[1]{\;\text{[}\unit[per-mode = symbol]{#1}\text{]}\;}
% 参照
\newcommand*{\fullref}[1]{\hyperref[{#1}]{Ref: \ref*{#1} \nameref*{#1}}} % https://tex.stackexchange.com/questions/121865/nameref-how-to-display-section-name-and-its-number

%%%
% 本文
%%%
\begin{document}

\title{熱力学メモ}
\author{NAKATA Keisuke}
\date{20240526}
\maketitle


\part*{熱力学}

\section{定数や単位の定義}

\subsection{アボガドロ定数 \texorpdfstring{$N_A$}{N\_A}}
\begin{align*}
  N_A \braunit{\per\mol}= \num{6.02214076e23}
\end{align*}
\par 0.012 kg の炭素に含まれる炭素原子の数. $N_A = \qty{1}{[\mol]}$
\cite[pp.1]{thermo}.

\subsection{大気圧 \texorpdfstring{$\qty{1}{[\atm]}$}{1[atm]}}
\begin{align*}
  \qty{1}{[\atm]} = \qty{1.01325e5}{[\Pa]}
\end{align*}
\par 海面上で空気から受ける圧力の値
\cite[pp.1]{thermo}.

\subsection{絶対温度 \texorpdfstring{$\unit{\K}$}{K}}
\begin{align*}
  T \braunit{\K} = t \braunit{\celsius} + 273.15
\end{align*}
\par \cite[pp.10]{thermo}

\section{状態方程式}

\subsection{シャルルの法則}
\begin{align*}
  \varDelta V\braunit{\m^3}
    &= \varDelta t\braunit{\celsius}\times\frac{V_{\qty{0}{\celsius}}\braunit{\m^3}}{273.15} \\
    &= \varDelta T\braunit{\K}\times\frac{V_{\qty{273.15}{\K}}\braunit{\m^3}}{273.15}
\end{align*}
\par 温度と体積の比例則. $V_{\qty{0}{\celsius}} (=V_{\qty{273.15}{\K}})$ は気体の種類や圧力, モル数によるので注意. 高圧または低温では分子間力や分子の大きさを無視できず, 近似が悪化する
\cite[pp.10]{thermo}.

\subsection{ボイル・シャルルの法則 (理想気体の状態方程式)}
\begin{align*}
  p\braunit{\Pa}V\braunit{\m^3} = n\braunit{\mol}RT\braunit{\K}
\end{align*}
\par $R=8.314\dots$ は気体定数. シャルルの法則, ボイルの法則, ゲイ=リュサックの法則を組み合わせたもの. 高圧または低温では分子間力や分子の大きさを無視できず, 近似が悪化する
\cite[pp.15]{thermo}.

\subsection{ファンデルワールスの状態方程式 (実在気体の状態方程式)}
\begin{align*}
  p\braunit{\Pa} = \frac{n\braunit{\mol}RT\braunit{\K}}{V\braunit{\m^3} - n\braunit{\mol}b} - a\left(\frac{n\braunit{\mol}}{V\braunit{\m^3}}\right)^2
\end{align*}
\begin{itemize}
  \item $a$ は分子間力による誤差を補正するパラメータ (分子間力は密度の2乗に比例する).
  \item $b$ は分子の大きさによる誤差を補正するパラメータ (\qty{1}{\mol} の気体の分子を寄せ集めてた時の体積で, 固体での体積とほぼ同じ).
\end{itemize}
\par \cite[pp.17]{thermo}

\subsection{ドルトンの法則}
理想混合系において, 複数の気体からなる混合気体のある温度での圧力 (\emph{全圧}) は,
それぞれの気体の同じ体積・同じ温度での圧力 (\emph{分圧}) の和に等しい
\cite[pp.18]{thermo}.

\section{熱, 仕事, 内部エネルギー}
\begin{itemize}
  \item $W\braunit{\J}$: 系が外界にする仕事
  \item $Q\braunit{\J}$: 系が外界から正味受け取る熱 (「高温熱源から熱 $Q_+$ を受け取り低温熱源に熱 $Q_-$ を放出する」場合, $Q = Q_+ - Q_-$)
  % \item $L\braunit{\J}$: 外界から系に加えられる力学的仕事 (ここでは $W$ とは区別する)
  \item $U\braunit{\J}$: 系の内部エネルギー
  \item $C\braunit{\J\per\K} = \dfrac{Q\braunit{\J}}{\varDelta T\braunit{\K}}$: 熱容量 (系の温度を $\varDelta T$ 上昇させるために必要な熱)
  \item $c\braunit{\J\per\mole\per\K} = \dfrac{C\braunit{\J\per\K}}{n\braunit{\mole}}$: モル比熱 ($1\braunit{\mole}$ あたりの熱容量)
\end{itemize}
\par 定積過程と定圧過程では, $\varDelta T$ が同じであっても到達する状態が異なるので, 熱容量・比熱の値は異なる.
定積過程の熱容量とモル比熱は $C_V, c_V$, 定圧過程の熱容量とモル比熱は $C_p, c_p$ と表記する
\cite[pp.38]{thermo}.

\subsection{内部エネルギー \texorpdfstring{$U\braunit{\J}$}{U[J]}}
\begin{align*}
  U = U(T, V, n) = \left[Q - W\right]_{OA} \braunit{\J}
\end{align*}
\par 暗に基準点 $U_0 = U(T_0, V_0, n) = 0$ を設定することで, 系の状態 $A = (T, V, n)$ における内部エネルギー $U$ は状態変数で決定できる
\cite[pp.65]{thermo}.

\subsubsection{理想気体}
\begin{align*}
  U\braunit{\J} = nc_VT + nu_0
\end{align*}
\par 理想気体の内部エネルギーは, 等温曲線上で一定値をとる.
つまり偏微分 ($T$ を変化させないように $V$ を変化させたときの $U$ の変化) $\pdv{U}{V} = 0$ である.
全微分については $\dv{U}{V} = \pdv{U}{V} + \pdv{U}{T}\dv{T}{V} = 0 + \pdv{U}{T}\dv{T}{V} \neq 0$ であることに注意
\cite[pp.67]{thermo}.

\subsubsection{実在気体}
\begin{align*}
  U\braunit{\J} = nc_VT -\frac{an^2}{V} + nu_0
\end{align*}
\par \cite[pp.92]{thermo}

\subsection{熱力学第1法則}
ある平衡状態 $A$ から別の平衡状態 $B$ への変化において, その変化の経路によらず,
\begin{align*}
  Q-W \braunit{\J} \quad \text{は一定.}
\end{align*}
ただし, 一般に $W \neq Q$ であることに注意
\cite[pp.36]{thermo}.
\par 内部エネルギーを用いて書くと
\begin{align*}
  & \left[Q - W\right]_{AB} = U_B - U_A = \int_A^B\left(\pdv{U}{T}\dd{T} + \pdv{U}{V}\dd{V}\right) = \Delta U_{AB} \braunit{\J} \\
  & \iff Q = \Delta U + W \braunit{\J}
\end{align*}
(上式) $Q-W$ は経路によらず一定なのだから, $\Delta U_{AB}$ は $A$ と $B$ の状態のみで決まる
\cite[pp.66]{thermo}.\\
(下式) 両辺の微小な変化を考えると
\begin{align*}
  \dd[\prime]{Q} = \dd{U} + \dd[\prime]{W} = \pdv{U}{T}\dd{T} + \left(\pdv{U}{V} + p\right)\dd{V} \braunit{\J}
\end{align*}
$\dd{U}$ は全微分だが, $\dd[\prime]{Q}, \dd[\prime]{W}$ は単に微小量でしかないことに注意
\cite[pp.75]{thermo}.

\subsection{\texorpdfstring{熱 $Q$ と仕事 $W$}{熱 Q と仕事 W }}
\begin{align*}
  Q\braunit{\J} &= \int_{T_1}^{T_2} C\left(T, V, n, \pdv{V}{T}\right)\dd{T} \\
  W\braunit{\J} &= \int_{V_1}^{V_2} p\left(T, V, n\right)\dd{V}
\end{align*}
\par 温度が $T_1$ から $T_2$ に変化する準静的過程で系が吸収する熱量 $Q$ は,
その過程中の温度 $T$ における熱容量 $C\left(T, V, n, \pdv{V}{T}\right)$ を $T$ で積分することで与えられる
\cite[pp.44]{thermo}.
\par 体積が $V_1$ から $V_2$ に変化する準静的過程で系が外界にする仕事 $W$ は,
その過程中の体積 $V$ における圧力 $p\left(T, V, n\right)$ を $V$ で積分することで与えられる.
なお, これは $pV$ 図での準静的過程の線の下側の面積にあたる.
過程がサイクルになっている場合,系が外界にする仕事は, サイクルが囲む面積にあたり, これは一般に正である
(系は外界から熱を受け取って外界に仕事をしながらもとの状態に戻る)
\cite[pp.48]{thermo}.

\subsubsection{サイクル} \label{section:熱と仕事_サイクル}
\begin{align*}
  W = Q\braunit{\J}
\end{align*}
\par サイクルとは, 同じ平衡状態へ戻ってくる過程である
\cite[pp.34]{thermo}.

\subsubsection{自由膨張} \label{section:熱と仕事_自由膨張}
\begin{align*}
  Q \overset{\mathrm{理想気体}}{=} W = 0\braunit{\J}
\end{align*}
\par 「$W = 0$」は, 自由膨張は真空への膨張なので抗力となる圧力がないため.
\par 「$Q = 0$」は, 理想気体では実験によって温度変化がないことが確かめられているため
\cite[pp.53]{thermo}.

\subsubsection{定圧過程}
\begin{align*}
  Q &= nc_p(T_2 - T_1)\braunit{\J} \\
  W &= \int_{V_1}^{V_2} p\left(T, V, n\right)\dd{V} = p(V_2 - V_1) \overset{\mathrm{理想気体}}{=} nR(T_2 - T_1) \braunit{\J}
\end{align*}
\par 「$= p(V_2 - V_1)$」は定圧過程 ($p$ が一定) より. 「$= nR(T_2 - T_1)$」は理想気体 ($p = \frac{nRT}{V}$) より
\cite[pp.48]{thermo}.

\subsubsection{等温過程}
\begin{align*}
  Q \overset{\mathrm{理想気体}}{=} W = \int_{V_1}^{V_2} p\left(T, V, n\right)\dd{V}
  \overset{\mathrm{理想気体}}{=} nRT\int_{V_1}^{V_2}\frac{1}{V}\dd{V} = nRT\ln\frac{V_2}{V_1} \braunit{\J}
\end{align*}
\par 「$= nRT\int_{V_1}^{V_2}\frac{1}{V}\dd{V}$」は等温過程 ($T$ が一定) かつ理想気体 ($p = \frac{nRT}{V}$) より
\cite[pp.49]{thermo}.
\par 「$W \overset{\mathrm{理想気体}}{=} Q$」について:
自由膨張 (理想気体では温度変化がない (\fullref{section:熱と仕事_自由膨張})) の状態変化を逆に戻す準静的等温過程をつなぎ合わせ,サイクルを作る.
\begin{itemize}
  \item サイクルでは $Q - W = 0$
  \item 自由膨張では $Q \overset{\mathrm{理想気体}}{=} W$
\end{itemize}
であるから,
$0 = \left[Q - W\right]_{サイクル} = \left[Q - W\right]_{自由膨張} + \left[Q - W\right]_{等温過程}
\overset{\mathrm{理想気体}}{=} 0 + \left[Q - W\right]_{等温過程}$
\cite[pp.56]{thermo}.

\subsubsection{定積過程}
\begin{align*}
  Q\braunit{\J} &= nc_V(T_2 - T_1) \\
  W\braunit{\J} &= 0
\end{align*}

\subsubsection{断熱過程}
\begin{align*}
  Q\braunit{\J} &= 0 \\
  W\braunit{\J} &\overset{\mathrm{理想気体}}{=} -nc_V(T_2 - T_1)
\end{align*}
\par この断熱過程が定積過程でなくても $c_V$ を使って良い.
\par 「$Q = 0$」は断熱の定義より.
\par 「$W = -nc_V(T_2 - T_1)$」について: $(p_1, V_1, T_1)$ を始点として,まず断熱過程で $(p^{\prime}, V_2, T_2)$ に変化させ,
次に定積過程で $(p^{\prime\prime}, V_2, T_1)$ に変化させ,
最後に等温過程で $(p_1, V_1, T_1)$ に変化させることでサイクルを作る.
\begin{itemize}
  \item サイクルでは $Q - W = 0$
  \item 断熱過程では $Q = 0$
  \item 定積過程では $W = 0, Q = nc_V(T_1 - T_2)$ (※ $T_2 \rightarrow T_1$ という変化であることに注意)
  \item 等温過程では $Q \overset{\mathrm{理想気体}}{=} W$
\end{itemize}
であるから,
$0 = \left[Q - W\right]_{サイクル} = \left[Q - W\right]_{断熱過程} + \left[Q - W\right]_{定積過程} + \left[Q - W\right]_{等温過程}
\overset{\mathrm{理想気体}}{=} - W_{断熱過程} + nc_V(T_1 - T_2)$
\cite[pp.61]{thermo}.

\subsection{ポアソンの法則}
断熱過程かつ理想気体において
\begin{align*}
  TV^\beta &= \text{const.} \\
  pV^\gamma &= \text{const.}
\end{align*}
ただし $\beta = \dfrac{R}{c_V}, \gamma = \dfrac{c_p}{c_V}$
\cite[pp.58]{thermo}.

\section{熱機関の効率}

\subsection{熱機関サイクルの効率 \texorpdfstring{$\eta$}{η}}
\begin{align*}
  \eta = \frac{W}{Q_+} = 1 - \frac{Q_-}{Q_+}
\end{align*}
\par $= 1 - \frac{Q_-}{Q_+}$ は (\fullref{section:熱と仕事_サイクル}) による
\cite[pp.78]{thermo}.

\subsection{熱力学第2法則}
「仕事を100\%熱に変換することができる (ジュールの実験)」が,
\emph{「高温熱源から吸収する熱 $Q_+$ を100\%仕事 $W$ に変換し, 低温熱源へ放出する熱 $Q_-$ をゼロにする」実験は知られていない.}
言い換えると, $\eta = 1$ を達成する熱機関は知られていない
\cite[pp.80]{thermo}.

\subsection{カルノーの定理} \label{section:熱機関の効率_カルノーの定理}
2つの熱源1, 2の間で働く熱機関において,
\begin{align*}
  \eta = 1 - \dfrac{Q_1}{Q_2} \leq 1 - \dfrac{T_1}{T_2} \quad \text{(可逆機関のとき等号成立)}
\end{align*}
\par \cite[pp.86]{thermo}, \cite[pp.93]{thermo}. つまり,
\begin{itemize}
  \item 可逆機関の効率が一般の熱機関の効率の上限 \cite[pp.82]{thermo}
  \item 可逆機関の効率は熱源の温度のみに依存し、可逆機関・作業物質の種類によらない \cite[pp.82]{thermo}
\end{itemize}

\subsection{クラウジウスの不等式}
サイクルにおいて, 温度 $T_i$ の熱源から熱量 $Q_i$ を正味吸収するとき,
\begin{align*}
  \sum_i \frac{Q_i}{T_i} \leq 0 \quad \text{(可逆機関のとき等号成立)}
\end{align*}
\par より一般に積分形式で書けば
\begin{align*}
  \oint \frac{\dd^\prime Q}{T} \leq 0 \quad \text{(可逆機関のとき等号成立)}
\end{align*}
\par \cite[pp.94]{thermo}

\bibliographystyle{jplain}
\bibliography{citations}

\end{document}
