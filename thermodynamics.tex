\documentclass[a4paper,11pt]{jsarticle}

%%%
% packages
%%%
% 数式
\usepackage{amsmath,amsfonts,physics}
\usepackage{bm}
% 画像
\usepackage[dvipdfmx]{graphicx}
% 単位
\usepackage{siunitx}
\AtBeginDocument{\RenewCommandCopy\qty\SI}  % qty が競合

%%%
% マクロ
%%%
% 太字
\newcommand{\bx}{\vb*{x}}
\newcommand{\bb}{\vb*{b}}
\newcommand{\bc}{\vb*{c}}
% 単位
\DeclareSIUnit{\atm}{atm}
\newcommand{\braunit}[1]{\;\text{[}\unit[per-mode = symbol]{#1}\text{]}\;}


%%%
% 本文
%%%
\begin{document}

\title{熱力学メモ}
\author{NAKATA Keisuke}
\date{20240526}
\maketitle


\part*{熱力学}

\subsection*{アボガドロ定数 $N_A$}
\begin{align*}
  N_A \braunit{\per\mol}= \num{6.02214076e23}
\end{align*}
0.012 kg の炭素に含まれる炭素原子の数。 $N_A = \qty{1}{[\mol]}$.
\cite[pp.1]{thermo}

\subsection*{大気圧 $\qty{1}{[\atm]}$}
\begin{align*}
  \qty{1}{[\atm]} = \qty{1.01325e5}{[\Pa]}
\end{align*}
海面上で空気から受ける圧力の値。
\cite[pp.1]{thermo}

\subsection*{絶対温度 $\unit{\K}$}
\begin{align*}
  T \braunit{\K} = t \braunit{\celsius} + 273.15
\end{align*}
\cite[pp.10]{thermo}

\subsection*{シャルルの法則}
\begin{align*}
  \varDelta V\braunit{\m^3}
    &= \varDelta t\braunit{\celsius}\times\frac{V_{\qty{0}{\celsius}}\braunit{\m^3}}{273.15} \\
    &= \varDelta T\braunit{\K}\times\frac{V_{\qty{273.15}{\K}}\braunit{\m^3}}{273.15}
\end{align*}
温度と体積の比例則。$V_{\qty{0}{\celsius}} (=V_{\qty{273.15}{\K}})$ は気体の種類や圧力、モル数によるので注意。高圧または低温では分子間力や分子の大きさを無視できず、近似が悪化する。
\cite[pp.10]{thermo}

\subsection*{ボイル・シャルルの法則 (理想気体の状態方程式)}
\begin{align*}
  p\braunit{\Pa}V\braunit{\m^3} = n\braunit{\mol}RT\braunit{\K}
\end{align*}
$R=8.314\dots$ は気体定数。シャルルの法則、ボイルの法則、ゲイ=リュサックの法則を組み合わせたもの。高圧または低温では分子間力や分子の大きさを無視できず、近似が悪化する。
\cite[pp.15]{thermo}

\subsection*{ファンデルワールスの状態方程式 (実在気体の状態方程式)}
\begin{align*}
  p\braunit{\Pa} = \frac{n\braunit{\mol}RT\braunit{\K}}{V\braunit{\m^3} - n\braunit{\mol}b} - a\left(\frac{n\braunit{\mol}}{V\braunit{\m^3}}\right)^2
\end{align*}
\begin{itemize}
  \item $a$ は分子間力による誤差を補正するパラメータ (分子間力は密度の2乗に比例する)。
  \item $b$ は分子の大きさによる誤差を補正するパラメータ (\qty{1}{\mol} の気体の分子を寄せ集めてた時の体積で、固体での体積とほぼ同じ)。
\end{itemize}
\cite[pp.17]{thermo}

\subsection*{ドルトンの法則}
理想混合系において、複数の気体からなる混合気体のある温度での圧力 (\emph{全圧}) は、
それぞれの気体の同じ体積・同じ温度での圧力 (\emph{分圧}) の和に等しい。
\cite[pp.18]{thermo}

\subsection*{熱と仕事の等価性}
\begin{itemize}
  \item $W\braunit{\J}$: 系が外界にする仕事
  \item $Q\braunit{\J}$: 系が外界から受け取る熱
  \item $L\braunit{\J}$: 外界から系に加えられる力学的仕事 (ここでは $W$ とは区別する)
\end{itemize}
サイクルにおいて、
\begin{align*}
  W\braunit{\J} = Q\braunit{\J}
\end{align*}
ただし、非サイクルにおいては、系の状態が変化するため、一般に $W \neq Q$ である。
\cite[pp.34]{thermo}

\subsection*{熱力学第1法則}
ある平衡状態から別の平衡状態へ変化するとき、 $Q-W$ という値はその経路によらず一定値となる。
\cite[pp.37]{thermo}

\subsection*{熱容量とモル比熱}
\begin{itemize}
  \item $C\braunit{\J\per\K} = \frac{Q\braunit{\J}}{\varDelta T\braunit{\K}}$: 熱容量
  \item $c\braunit{\J\per\mole\per\K} = \frac{C\braunit{\J\per\K}}{n\braunit{\mole}}$: モル比熱
\end{itemize}
熱容量は、系の温度を $\varDelta T$ だけ上昇させるために必要な熱 $Q$ を決定する比例係数。\\
モル比熱は、 $1\braunit{\mole}$ あたりの熱容量。\\
定積過程と定圧過程では、 $\varDelta T$ が同じであっても到達する状態が異なるので、熱容量・比熱の値は異なる。
定積過程の熱容量とモル比熱は $C_V, c_V$ 、定圧過程の熱容量とモル比熱は $C_p, c_p$ と表記する。
\cite[pp.38]{thermo}

\subsection*{外界にする仕事}
\begin{align*}
  W\braunit{\J} = p_e\braunit{\Pa}\varDelta V\braunit{\m^3}
\end{align*}
\begin{itemize}
  \item $W$: 系が外界に対してする仕事
  \item $p_e$: 外界から系に作用する圧力
  \item $\varDelta V$: 系の体積の変化
\end{itemize}
系が $p_e$ に抗して $\varDelta V$ だけ膨張したときに外界にする仕事。\\
これは、仕事が力と距離の積であることと、圧力が単位面積あたりの力であることから導ける。
\cite[pp.46]{thermo}

\subsection*{準静的過程の吸熱}
\begin{align*}
  Q\braunit{\J} = \int_{T_1}^{T_2} C\left(T, V, n, \pdv{V}{T}\right)\dd{T}
\end{align*}
温度が $T_1$ から $T_2$ に変化する準静的過程で系が吸収する熱量 $Q$ は、
その過程中の温度 $T$ における熱容量 $C\left(T, V, n, \pdv{V}{T}\right)$ を $T$ で積分することで与えられる。
\cite[pp.44]{thermo}

\subsection*{準静的過程の仕事}
\begin{align*}
  W\braunit{\J} = \int_{V_1}^{V_2} p\left(T, V, n\right)\dd{V}
\end{align*}
体積が $V_1$ から $V_2$ に変化する準静的過程で系が外界にする仕事 $W$ は、
その過程中の体積 $V$ における圧力 $p\left(T, V, n\right)$ を $V$ で積分することで与えられる。\\
なお、これは $pV$ 図での準静的過程の線の下側の面積にあたる。
過程がサイクルになっている場合、系が外界にする仕事は、サイクルが囲む面積にあたり、これは一般に正である
(系は熱を受け取って外界に仕事をしながらもとの状態に戻る)。

ここで、定圧過程 ($p$ が一定) かつ理想気体 ($p = \frac{nRT}{V}$) を考えると、
\begin{align*}
  W\braunit{\J} = p(V_2 - V_1) = nR(T_2 - T_1)
\end{align*}
となる。つまり、定圧過程で理想気体のする仕事は、圧力と体積差の積、もしくは物質量と温度差 (と気体定数) の積となる。

また、定温過程 ($T$ が一定) かつ理想気体 ($p = \frac{nRT}{V}$) を考えると、
\begin{align*}
  W\braunit{\J} = nRT\int_{V_1}^{V_2}\frac{1}{V}\dd{V} = nRT\ln\frac{V_2}{V_1}
\end{align*}
となる。つまり、定温過程で理想気体のする仕事は、温度と対数体積比の積となる。
\cite[pp.48]{thermo}

\bibliographystyle{jplain}
\bibliography{citations}

\end{document}
